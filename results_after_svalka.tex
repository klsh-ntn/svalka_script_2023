% This is a generated file, do not edit
\documentclass[12pt]{article}
\usepackage[a4paper,top=1cm,bottom=1cm]{geometry}

\usepackage{graphicx}
\usepackage{color}

\usepackage[utf8]{inputenc}
\usepackage[T2A]{fontenc}
\usepackage[russian]{babel}

\begin{document}

\pagestyle{empty}

\begin{center}
\begin{tabular}{cc}
\includegraphics[scale=0.48]{klshlogo.pdf} &
\raisebox{0.5cm}{
  \begin{tabular}{c}
    {\Large\bf КРАСНОЯРСКАЯ ЛЕТНЯЯ ШКОЛА}\\
    \ \\
    {\large\bf Сезон 2022 года}\\
   \end{tabular}
}
\end{tabular}
\ \\
\ \\
\ \\
{\Large\bf Результаты физико--математического турнира}
\ \\
\end{center}

Команды расположены в таблице в порядке убывания итогового рейтинга. 
{\em Рейтингом} команды называется количество команд, выступивших
менее успешно. Итоговый рейтинг есть сумма рейтингов в основном туре и
свалке. В случае, когда две команды имеют одинаковую сумму рейтингов в
основном туре и свалке, более успешной считается команда, чей рейтинг
в основном туре выше.

Команды, занимающие \textcolor{red}{первое и второе место} в итоговом
рейтинге, встречаются в суперфинале за абсолютное первое
место. Команды, занимающие \textcolor{green}{третье и четвёртое
место}, встречаются за абсолютное третье место.

\vspace*{1cm}

\begin{center}
    \begin{tabular}{|c|c|c|c|}

      \hline
      {\bf Команда} & {\bf Основной тур} &
      {\bf Финальная свалка} & {\bf Итоговый рейтинг} \\

      \hline
      \hline
\textcolor{red}{$\pi$} & 
\textcolor{red}{21} & 
\textcolor{red}{21} & 
\textcolor{red}{42} \\ 
\hline
\textcolor{red}{$\gamma$} & 
\textcolor{red}{20} & 
\textcolor{red}{22} & 
\textcolor{red}{42} \\ 
\hline
\textcolor{green}{$\eta$} & 
\textcolor{green}{23} & 
\textcolor{green}{18} & 
\textcolor{green}{41} \\ 
\hline
\textcolor{green}{$\varphi$} & 
\textcolor{green}{16} & 
\textcolor{green}{23} & 
\textcolor{green}{39} \\ 
\hline
\textcolor{black}{$\alpha$} & 
\textcolor{black}{22} & 
\textcolor{black}{15} & 
\textcolor{black}{37} \\ 
\hline
\textcolor{black}{$\chi$} & 
\textcolor{black}{12} & 
\textcolor{black}{20} & 
\textcolor{black}{32} \\ 
\hline
\textcolor{black}{$\beta$} & 
\textcolor{black}{15} & 
\textcolor{black}{16} & 
\textcolor{black}{31} \\ 
\hline
\textcolor{black}{$\kappa$} & 
\textcolor{black}{19} & 
\textcolor{black}{10} & 
\textcolor{black}{29} \\ 
\hline
\textcolor{black}{$\iota$} & 
\textcolor{black}{10} & 
\textcolor{black}{19} & 
\textcolor{black}{29} \\ 
\hline
\textcolor{black}{$\omega$} & 
\textcolor{black}{11} & 
\textcolor{black}{17} & 
\textcolor{black}{28} \\ 
\hline
\textcolor{black}{$\nu$} & 
\textcolor{black}{14} & 
\textcolor{black}{13} & 
\textcolor{black}{27} \\ 
\hline
\textcolor{black}{$\delta$} & 
\textcolor{black}{17} & 
\textcolor{black}{9} & 
\textcolor{black}{26} \\ 
\hline
\textcolor{black}{$\zeta$} & 
\textcolor{black}{13} & 
\textcolor{black}{8} & 
\textcolor{black}{21} \\ 
\hline
\textcolor{black}{$\upsilon$} & 
\textcolor{black}{18} & 
\textcolor{black}{0} & 
\textcolor{black}{18} \\ 
\hline
\textcolor{black}{$o$} & 
\textcolor{black}{5} & 
\textcolor{black}{11} & 
\textcolor{black}{16} \\ 
\hline
\textcolor{black}{$\varepsilon$} & 
\textcolor{black}{2} & 
\textcolor{black}{14} & 
\textcolor{black}{16} \\ 
\hline
\textcolor{black}{$\psi$} & 
\textcolor{black}{3} & 
\textcolor{black}{12} & 
\textcolor{black}{15} \\ 
\hline
\textcolor{black}{$\theta$} & 
\textcolor{black}{9} & 
\textcolor{black}{4} & 
\textcolor{black}{13} \\ 
\hline
\textcolor{black}{$\rho$} & 
\textcolor{black}{7} & 
\textcolor{black}{5} & 
\textcolor{black}{12} \\ 
\hline
\textcolor{black}{$\tau$} & 
\textcolor{black}{6} & 
\textcolor{black}{6} & 
\textcolor{black}{12} \\ 
\hline
\textcolor{black}{$\xi$} & 
\textcolor{black}{4} & 
\textcolor{black}{7} & 
\textcolor{black}{11} \\ 
\hline
\textcolor{black}{$\lambda$} & 
\textcolor{black}{8} & 
\textcolor{black}{0} & 
\textcolor{black}{8} \\ 
\hline
\textcolor{black}{$\mu$} & 
\textcolor{black}{1} & 
\textcolor{black}{0} & 
\textcolor{black}{1} \\ 
\hline
\textcolor{black}{$\sigma$} & 
\textcolor{black}{0} & 
\textcolor{black}{0} & 
\textcolor{black}{0} \\ 
\hline
\hline

    \end{tabular}

  \end{center}

\end{document}
